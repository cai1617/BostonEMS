\section*{Abstract}
The Boston Emergency Medical Services, has provided 142,341 support units including basic life support (\textbf{BLS}) and advanced life support (\textbf{ALS}) in 2013. Surprisingly enough, there were only 24 frontline ambulances serving a population of about 645,000. It indicates that a shift schedule has a great impact on the efficiency of Boston EMS (City of Boston). 
Therefore, we would like to investigate how a shift schedule will affect the system performance of Boston EMS. More specifically, the goal of this proposed project is to optimize a shift schedule for each station as well as to find the most optimal way to place paramedics and ambulances at stations located in and around the Boston metropolitan area.
 The 2013 Boston EMS data will help us determine which station we would like to focus on initially. 

%%%This will be the ABSTRACT
We present a method to optimize a shift schedule for each station as well as finding the most optimal way to place paramedics and ambulances at stations located in and around the Boston metropolitan area. The goal is to minimize the need for ambulances during a specific time period, given a certain expected coverage. Using data received from the Boston Emergency Services regarding the number of incidents per hour in a day, we generated a demand curve for our study area Roxbury. We plan to use the model of the study area to analyze other parts of Boston. The end goal is to compare our model with the linear model in Ragopalan et. al paper Ambulance Deployment and Shift Scheduling: An Integrated Approach. Our final result shows that

\subsection*{Intellectual Merit}
 
This proposed project is significant because it�s necessary to improve the quality of the Boston EMS. Although they bring high quality to the EMS field, it is always better to improve the system. Mathematically, the purpose of this project is to find the most optimal way to place the 244 EMTs, 64 paramedics, and the ambulances to make it more efficient for them to get to the scene. Our proposed model will be similar to that discussed in Church's article, Manpower Deployment in Emergency Systems but it is slightly different because they focus on one station where in our optimized model we want to incorporate more than one station. Although this problem has been tackled before, it is important because urban planning for ambulances and paramedics are essential to providing efficient emergency services (Church). We hope to utilize operation research techniques such as integer programming and linear programming to solve our model. Once our finalized model is complete, the goal is to simulate an emergency using R that will test multiple stations. From this we will see how well our staffing model will perform.

\subsection*{Broader Impacts}

The broader impact of this project is that it can be applied to various other emergency services such as fire departments, police departments as well as food delivery services, and UBER services. In the case if the fire and police departments, our model can be applied to them because it�s almost similar. Instead of having ambulances, you can change it to fire engines and police cruisers, and police officers and firefighters instead of EMTs and paramedics. And UBER is like a taxi service and the model can be applied in this instance because people call them for rides and generally wait for the drivers to pick them up. The model can still be applied because the UBER drivers would have to reach the person in a set amount of time.